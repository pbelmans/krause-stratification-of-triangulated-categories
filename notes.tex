\documentclass[10pt,a4paper]{article}
\usepackage{hyperref}
\usepackage{cleveref}
\hypersetup{hypertexnames = false, bookmarksdepth = 2, bookmarksopen = true, colorlinks, linkcolor = black, citecolor = black, urlcolor = black, pdfstartview={XYZ null null 1}}

\usepackage{amsfonts}
\usepackage[fleqn, leqno]{amsmath}
\usepackage{amsthm}
\usepackage{biblatex}
\usepackage{booktabs}
\usepackage{diagbox}
\usepackage{enumitem}
\usepackage{fixltx2e}
\usepackage{mathtools}
\usepackage{thmtools}
\usepackage{tikz-cd}
\usepackage[colorinlistoftodos]{todonotes}
\usepackage{xparse}
\usepackage{xspace}

\usepackage[T1]{fontenc}
\usepackage[charter]{mathdesign}
\usepackage[scaled]{beramono,berasans}
\usepackage{eucal}
\usepackage{epstopdf}
\usepackage{microtype}
\frenchspacing


\addbibresource{bibliography.bib}

\relpenalty=10000
\binoppenalty=10000

% todonotes configuration
\newcounter{todocounter}
\DeclareDocumentCommand\addreference{g}{\stepcounter{todocounter}\todo[color = blue!30, fancyline]{\thetodocounter. Add reference\IfNoValueF{#1}{: #1}}\xspace}
\DeclareDocumentCommand\checkthis{g}{\stepcounter{todocounter}\todo[color = red!50, fancyline]{\thetodocounter. Check this\IfNoValueF{#1}{: #1}}\xspace}
\DeclareDocumentCommand\fixthis{g}{\stepcounter{todocounter}\todo[color = orange!50, fancyline]{\thetodocounter. Fix this\IfNoValueF{#1}{: #1}}\xspace}
\DeclareDocumentCommand\expand{g}{\stepcounter{todocounter}\todo[color = green!50, fancyline]{\thetodocounter. Expand\IfNoValueF{#1}{: #1}}\xspace}
\newcommand\removethis{\stepcounter{todocounter}\todo[color=yellow!50]{\thetodocounter. Remove this?}}

% environments
\declaretheoremstyle[
  spaceabove = 3pt,
  spacebelow = 3pt,
]{lecture}
\theoremstyle{lecture}
\newtheorem{theorem}{Theorem}
\newtheorem{corollary}[theorem]{Corollary}
\newtheorem{definition}[theorem]{Definition}
\newtheorem{example}[theorem]{Example}
\newtheorem{exercise}[theorem]{Exercise}
\newtheorem{lemma}[theorem]{Lemma}
\newtheorem{proposition}[theorem]{Proposition}
\newtheorem{remark}[theorem]{Remark}


\mathchardef\mhyphen="2D
\newcommand\dash{\nobreakdash-\hspace{0pt}}

\newcommand\Ab{\ensuremath{\mathrm{Ab}}}
\newcommand\ac{\ensuremath{\mathrm{ac}}}
\newcommand\bounded{\ensuremath{\mathrm{b}}}
\newcommand\cc{\ensuremath{\mathrm{c}}}
\newcommand\derived{\ensuremath{\mathbf{D}}}
\newcommand\identity{\ensuremath{\mathrm{id}}}
\newcommand\Mod{\ensuremath{\mathrm{Mod}}}
\newcommand\opp{\ensuremath{\mathrm{op}}}

\DeclareMathOperator\cone{cone}
\DeclareMathOperator\Hom{Hom}
\DeclareMathOperator\image{image}
\DeclareMathOperator\Loc{Loc}
\DeclareMathOperator\Mor{Mor}
\DeclareMathOperator\Thick{Thick}


\title{Stratification of triangulated categories}
\author{Pieter Belmans}

\begin{document}
\maketitle

\tableofcontents

\section{Day 2: Infinite methods}
\subsection{Compact objects}
Let~$\mathcal{T}$ be a triangulated category, denote~$\Sigma$ its shift or suspension, and assume that~$\mathcal{T}$ has set-indexed coproducts.
\begin{definition}
  An object~$X$ of~$\mathcal{T}$ is \emph{compact} if~$\Hom_{\mathcal{T}}(X,-)\colon\mathcal{T}\to\Ab$ preserves coproducts.
\end{definition}
\begin{lemma}
  An object~$X$ is compact if and only if for all~$X\to\coprod_{i\in I}Y_i$ there exists a factorisation through~$\coprod_{i\in I}Y_i$, where~$I_0\subseteq I_0$ is a finite subset.
\end{lemma}
\begin{remark}
  These compact objects serve as building blocks for the category~$\mathcal{T}$ and they constitute a thick subcategory~$\mathcal{T}^\cc$.
\end{remark}
\begin{definition}
  The category~$\mathcal{T}$ is \emph{compactly generated} if~$\mathcal{T}=\Loc(\mathcal{C})$, for some set of compact objects~$\mathcal{C}$.
\end{definition}
\begin{proposition}
  For a set of compact objects~$\mathcal{C}\subseteq\mathcal{T}^\cc$ the following are equivalent:
  \begin{enumerate}
    \item $\Loc(\mathcal{C})=\mathcal{T}$;
    \item for all objects~$X$ in~$\mathcal{T}$ such that~$\Hom_{\mathcal{T}}(\Sigma^nC,X)=0$ for all objects~$C\in\mathcal{C}$ and~$n\in\mathbb{Z}$ we have~$X=0$.
  \end{enumerate}

  \begin{proof}
    From (1) to (2) is easy: let~$X$ be an object of~$\mathcal{T}$ and consider
    \begin{equation}
      % TODO improve notation
      {}^\perp X\coloneqq\left\{ V\in\mathcal{T}\mid\Hom_{\mathcal{T}}(\Sigma^nV,X)=0,\forall n\in\mathbb{Z} \right\}
    \end{equation}
    which is a localising subcategory of~$\mathcal{T}$. If~$\mathcal{C}\subseteq^\perp X$ then~$X=0$.

    From (2) to (1) we have to use Brown representability (see later), which depends on the compactness of the objects. The inclusion~$\Loc(\mathcal{C})\hookrightarrow\mathcal{T}$ has a right adjoint~$\Gamma$. Let~$X$ be an object of~$\mathcal{T}$, by the adjunction we have a morphism~$\Gamma(X)\to X$ that we can complete to a triangle
    \begin{equation}
      \Gamma(X)\to X\to X'\to
    \end{equation}
    and the long exact sequence obtained by applying~$\Hom_{\mathcal{T}}(V,-)$ tells us that~$\Hom_{\mathcal{T}}(V,X')=0$ for all~$X\in\Loc(\mathcal{C})$. So if (2) holds then we have~$X'=0$ and therefore~$X\in\Loc(\mathcal{C})$.% TODO improve the proof
  \end{proof}
\end{proposition}
\begin{proposition}
  Let~$X$ and~$Y$ be compact objects of~$\mathcal{T}$. If~$Y\in\Loc(X)$ then~$Y\in\Thick(X)$.
\end{proposition}
Hence if a compact objects~$Y$ can be obtained from a compact objects~$X$ in infinitely many steps it can also be done in finitely many.
\begin{example}
  Let~$A$ be any ring and take~$X\in\derived(\Mod/A)$. Then the following are equivalent:
  \begin{enumerate}
    \item $X\in\Thick(A)$;
    \item $X$ is isomorphic to a perfect complex;
    \item $X$ is compact.
  \end{enumerate}
\end{example}

\subsection{Brown representability}
\begin{theorem}
  Let~$\mathcal{T}$ be a compactly generated triangulated category. For a functor~$H\colon\mathcal{T}^\opp\to\Ab$ the following are equivalent:
  \begin{enumerate}
    \item $H$ is cohomological (i.e.\ it sends exact triangles to exact sequences and it sends coproducts to products);
    \item $H$ is representable (i.e.\ there exists an object~$X\in\mathcal{T}$ such that~$H\cong\Hom_{\mathcal{T}}(-,X)$).
  \end{enumerate}
\end{theorem}
\begin{corollary}
  Let~$\mathcal{T}$ be a compactly generated triangulated category. Then~$\mathcal{T}$ has all coproducts.
  \begin{proof}
    Consider~$\{X_i\}_{i\in I}$ a family of objects in~$\mathcal{T}$. Then~$\prod_{i\in I}\Hom_{\mathcal{T}}(-,X_i)$ satisfies condition (1) in the Brown representability theorem, hence it must be representable by an object that satisfies the universal property for a product.
  \end{proof}
\end{corollary}
\begin{corollary}
  Let~$\mathcal{T}$ be a compactly generated triangulated category and~$\mathcal{U}$ be any triangulated category. Then for all exact functors~$F\colon\mathcal{T}\to\mathcal{U}$ the following are equivalent:
  \begin{enumerate}
    \item $F$ preserves coproducts;
    \item $F$ admits a right adjoint~$G$.
  \end{enumerate}
  \begin{proof}
    From (1) to (2) we consider any object~$X$ in~$\mathcal{U}$, then~$\Hom_{\mathcal{U}}(F(-),X)\colon\mathcal{T}^\opp\to\Ab$ satisfies condition (1) in the Brown representability theorem, hence it can be represented as~$\Hom_{\mathcal{T}}(-,Y)$, and set~$G(X)\coloneqq Y$.

    From (2) to (1) is trivial as left adjoints preserve coproducts.
  \end{proof}
\end{corollary}
We will often take~$\mathcal{U}$ a Verdier quotient in applications.

\subsection{Bousfield localisation}
Let~$\mathcal{T}$ be a triangulated category and~$\mathcal{S}$ a thick subcategory. Its \emph{Verdier quotient}
\begin{equation}
  \mathcal{T}/\mathcal{S}\coloneqq\mathcal{T}[\{\sigma\in\Mor(\mathcal{T})\mid\cone(\sigma)\in\mathcal{S}\}^{-1}]
\end{equation}
is again triangulated, the quotient functor~$Q\colon\mathcal{T}\to\mathcal{S}$ is exact and its kernel~$\ker(Q)=\mathcal{S}$, where the \emph{kernel} and \emph{essential image}
\begin{equation}
  \begin{aligned}
    \ker(F)&\coloneqq\{X\in\mathcal{T}\mid F(X)=0\} \\
    \image(F)&\coloneqq\{Y\in\mathcal{U}\mid \exists X\in\mathcal{T}\colon Y\cong F(X)\}
  \end{aligned}
\end{equation}
are full triangulated subcategories.
\begin{example}
  Let~$\mathcal{A}$ be an abelian category. Then~$\derived(\mathcal{A})\coloneqq\mathbf{K}(\mathcal{A})/\mathbf{K}_\ac(\mathcal{A})$ is a triangulated category.
\end{example}
\begin{remark}
  A priori it is unclear whether a Verdier quotient~$\mathcal{T}/\mathcal{S}$ is locally small.
\end{remark}
\begin{definition}
   there is a natural transformation~$\eta\colon\identity_{\mathcal{T}}\Rightarrow L$ such that
  \begin{enumerate}
    \item $L\circ\eta\colon L\Rightarrow L^2$ is invertible, i.e.\ for all~$X$ the morphism~$L(\eta_X)\colon L(X)\to L^2(X)$ is an isomorphism;
    \item $L\circ\eta=\eta\circ L$, i.e.\ $L(\eta_X)=\eta_{L(X)}$.
  \end{enumerate}
\end{definition}
\begin{definition}
  A \emph{localisation functor} is an exact endofunctor~$\Gamma\colon\mathcal{T}\to\mathcal{T}$ for which~$\Gamma^\opp$ is a localisation functor.
\end{definition}
\begin{remark}
  The notation~$\Gamma$ is a reference both to Grothendieck's local cohomology functor and the symmetry between~$L$ and~$\Gamma$.
\end{remark}
\begin{proposition}
  For a thick subcategory~$\mathcal{S}$ the following are equivalent:
  \begin{enumerate}
    \item $\mathcal{S}\hookrightarrow\mathcal{T}$ admits a right adjoint;
    \item $Q\colon\mathcal{T}\to\mathcal{T}/\mathcal{S}$ admits a right adjoint;% TODO check these
    \item there exists a localisation functor~$L$ such that~$\ker(L)=\mathcal{S}$;
    \item there exists a colocalisation functor~$\Gamma$ such that~$\image(\Gamma)=\mathcal{S}$.
  \end{enumerate}
\end{proposition}
In this case, the following holds:
\begin{enumerate}
  \item for all objects~$X$ in~$\mathcal{T}$ there exists a functorial triangle
    \begin{equation}
      \Gamma(X)\to X\to L(X)\to
    \end{equation}
  \item $\mathcal{S}^\perp=\image(L)=\ker(\Gamma)$ and~${}^\perp(\mathcal{S}^\perp)=\mathcal{S}$ (where a priori the left-hand side could be bigger);
  \item $\Gamma$ induces a right adjoint for the inclusion~$\mathcal{S}\hookrightarrow\mathcal{T}$;
  \item $L$ induces a left adjoint for the inclusion~$\mathcal{S}^\perp\hookrightarrow\mathcal{T}$;
\end{enumerate}
\begin{remark}
  This is related to the notion of semi-orthogonal decompositions: we obtain~$\mathcal{T}=\langle\mathcal{S}^\perp,\mathcal{S}\rangle=\langle\mathcal{S},^\perp\mathcal{S}\rangle$. A useful mnemonic (suggested by Sasha Kuznetsov) is that~$\perp$ is always in the middle.
\end{remark}
\begin{example}
  Let~$\mathcal{T}$ be a compactly generated triangulated category. Let~$\mathcal{S}=\Loc(\mathcal{C})$ be a localising subcategory generated by a set of compact objects~$\mathcal{C}$. We obtain a diagram% TODO fix double headed
  \begin{equation}
    \begin{tikzcd}
      \mathcal{S}^\cc \arrow[hook]{r} \arrow[hook]{d} & \mathcal{T}^\cc \arrow{r} \arrow[hook]{d} & \mathcal{T}^\cc/\mathcal{S}^\cc \arrow{d} \\
      \mathcal{S} \arrow[hook]{r} & \mathcal{T} \arrow{r} & \mathcal{T}/\mathcal{S}
    \end{tikzcd}
  \end{equation}
  where~$\mathcal{S}^\cc=\Thick(\mathcal{C})$ and the induced functor~$\mathcal{T}^\cc/\mathcal{S}^\cc\to(\mathcal{T}/\mathcal{S})^\cc$ is an equivalence up to direct summands.
\end{example}

\subsection{Describing localising subcategories}
The problem that we would like to solve is the description of all localising subcategories of a compactly generated triangulated category. Some obvious questions that arise are:
\begin{enumerate}
  \item do they form a set or a proper class?
  \item how can we explicitly describe them?
\end{enumerate}

% TODO finish this last part


\section{Day 3: Stratification of big triangulated categories}
\subsection{Rings acting on triangulated categories}

\subsection{Stratification of big triangulated categories}

\subsection{Consequences of stratification}


\section{Day 4: $\mathbf{K}(\Inj/X)$ and Grothendieck duality}
\subsection{$\derived^\bounded(X)$ and compact objects}

\subsection{Grothendieck duality}


\section{Day 5: Stratification of small triangulated categories}
\subsection{Example: stratifying the bounded derived category of the Kronecker algebra}

\subsection{Stratification of bounded derived categories of hereditary algebras}

\subsection{Stratification of small triangulated categories}

\end{document}

