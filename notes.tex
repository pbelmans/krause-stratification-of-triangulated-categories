\documentclass[10pt,a4paper]{article}
\usepackage{hyperref}
\usepackage{cleveref}
\hypersetup{hypertexnames = false, bookmarksdepth = 2, bookmarksopen = true, colorlinks, linkcolor = black, citecolor = black, urlcolor = black, pdfstartview={XYZ null null 1}}

\usepackage{amsfonts}
\usepackage[fleqn, leqno]{amsmath}
\usepackage{amsthm}
\usepackage{biblatex}
\usepackage{booktabs}
\usepackage{diagbox}
\usepackage{enumitem}
\usepackage{fixltx2e}
\usepackage{mathtools}
\usepackage{thmtools}
\usepackage{tikz-cd}
\usepackage[colorinlistoftodos]{todonotes}
\usepackage{xparse}
\usepackage{xspace}

\usepackage[T1]{fontenc}
\usepackage[charter]{mathdesign}
\usepackage[scaled]{beramono,berasans}
\usepackage{eucal}
\usepackage{epstopdf}
\usepackage{microtype}
\frenchspacing


\addbibresource{bibliography.bib}

\relpenalty=10000
\binoppenalty=10000

% todonotes configuration
\newcounter{todocounter}
\DeclareDocumentCommand\addreference{g}{\stepcounter{todocounter}\todo[color = blue!30, fancyline]{\thetodocounter. Add reference\IfNoValueF{#1}{: #1}}\xspace}
\DeclareDocumentCommand\checkthis{g}{\stepcounter{todocounter}\todo[color = red!50, fancyline]{\thetodocounter. Check this\IfNoValueF{#1}{: #1}}\xspace}
\DeclareDocumentCommand\fixthis{g}{\stepcounter{todocounter}\todo[color = orange!50, fancyline]{\thetodocounter. Fix this\IfNoValueF{#1}{: #1}}\xspace}
\DeclareDocumentCommand\expand{g}{\stepcounter{todocounter}\todo[color = green!50, fancyline]{\thetodocounter. Expand\IfNoValueF{#1}{: #1}}\xspace}
\newcommand\removethis{\stepcounter{todocounter}\todo[color=yellow!50]{\thetodocounter. Remove this?}}

% environments
\declaretheoremstyle[
  spaceabove = 3pt,
  spacebelow = 3pt,
]{lecture}
\theoremstyle{lecture}
\newtheorem{theorem}{Theorem}
\newtheorem{corollary}[theorem]{Corollary}
\newtheorem{definition}[theorem]{Definition}
\newtheorem{example}[theorem]{Example}
\newtheorem{exercise}[theorem]{Exercise}
\newtheorem{lemma}[theorem]{Lemma}
\newtheorem{proposition}[theorem]{Proposition}
\newtheorem{remark}[theorem]{Remark}


\mathchardef\mhyphen="2D
\newcommand\dash{\nobreakdash-\hspace{0pt}}

\newcommand\Ab{\ensuremath{\mathrm{Ab}}}
\newcommand\ac{\ensuremath{\mathrm{ac}}}
\newcommand\bounded{\ensuremath{\mathrm{b}}}
\newcommand\cc{\ensuremath{\mathrm{c}}}
\newcommand\derived{\ensuremath{\mathbf{D}}}
\newcommand\identity{\ensuremath{\mathrm{id}}}
\newcommand\Mod{\ensuremath{\mathrm{Mod}}}
\newcommand\opp{\ensuremath{\mathrm{op}}}

\DeclareMathOperator\cone{cone}
\DeclareMathOperator\Hom{Hom}
\DeclareMathOperator\image{image}
\DeclareMathOperator\Loc{Loc}
\DeclareMathOperator\Mor{Mor}
\DeclareMathOperator\Thick{Thick}


\title{Stratification of triangulated categories}
\author{Pieter Belmans}

\begin{document}
\maketitle

\tableofcontents

\clearpage

\section{Day 1: Support and cosupport of complexes}

\section{Day 2: Infinite methods}
\subsection{Compact objects}
Let~$\mathcal{T}$ be a triangulated category, denote~$\Sigma$ its shift or suspension, and assume that~$\mathcal{T}$ has set-indexed coproducts.
\begin{definition}
  An object~$X$ of~$\mathcal{T}$ is \emph{compact} if~$\Hom_{\mathcal{T}}(X,-)\colon\mathcal{T}\to\Ab$ preserves coproducts.
\end{definition}
\begin{lemma}
  An object~$X$ is compact if and only if for all~$X\to\coprod_{i\in I}Y_i$ there exists a factorisation through~$\coprod_{i\in I}Y_i$, where~$I_0\subseteq I_0$ is a finite subset.
\end{lemma}
\begin{remark}
  These compact objects serve as building blocks for the category~$\mathcal{T}$ and they constitute a thick subcategory~$\mathcal{T}^\cc$.
\end{remark}
\begin{definition}
  The category~$\mathcal{T}$ is \emph{compactly generated} if~$\mathcal{T}=\Loc(\mathcal{C})$, for some set of compact objects~$\mathcal{C}$.
\end{definition}
\begin{proposition}
  For a set of compact objects~$\mathcal{C}\subseteq\mathcal{T}^\cc$ the following are equivalent:
  \begin{enumerate}
    \item $\Loc(\mathcal{C})=\mathcal{T}$;
    \item for all objects~$X$ in~$\mathcal{T}$ such that~$\Hom_{\mathcal{T}}(\Sigma^nC,X)=0$ for all objects~$C\in\mathcal{C}$ and~$n\in\mathbb{Z}$ we have~$X=0$.
  \end{enumerate}

  \begin{proof}
    From (1) to (2) is easy: let~$X$ be an object of~$\mathcal{T}$ and consider
    \begin{equation}
      \prescript{\perp}{}{X}\coloneqq\left\{ V\in\mathcal{T}\mid\Hom_{\mathcal{T}}(\Sigma^nV,X)=0,\forall n\in\mathbb{Z} \right\}
    \end{equation}
    which is a localising subcategory of~$\mathcal{T}$. If~$\mathcal{C}\subseteq\prescript{\perp}{}{X}$ then~$X=0$.

    From (2) to (1) we have to use Brown representability (see later), which depends on the compactness of the objects. The inclusion~$\Loc(\mathcal{C})\hookrightarrow\mathcal{T}$ has a right adjoint~$\Gamma$. Let~$X$ be an object of~$\mathcal{T}$, by the adjunction we have a morphism~$\Gamma(X)\to X$ that we can complete to a triangle
    \begin{equation}
      \Gamma(X)\to X\to X'\to
    \end{equation}
    and the long exact sequence that we can obtain by applying~$\Hom_{\mathcal{T}}(V,-)$ tells us that~$\Hom_{\mathcal{T}}(V,X')=0$ for all~$X\in\Loc(\mathcal{C})$. So if (2) holds then we have~$X'=0$ and therefore~$X\in\Loc(\mathcal{C})$.% TODO improve the proof
  \end{proof}
\end{proposition}
\begin{proposition}
  Let~$X$ and~$Y$ be compact object of~$\mathcal{T}$. If~$Y\in\Loc(X)$ then we already have~$Y\in\Thick(X)$.
\end{proposition}
Hence if a compact objects~$Y$ can be obtained from a compact objects~$X$ in infinitely many steps it can also be done in finitely many.
\begin{example}
  Let~$A$ be any ring and take~$X\in\derived(\Mod/A)$. Then the following are equivalent:
  \begin{enumerate}
    \item $X\in\Thick(A)$;
    \item $X$ is isomorphic to a perfect complex;
    \item $X$ is compact.
  \end{enumerate}
\end{example}

\subsection{Brown representability}
\begin{theorem}
  Let~$\mathcal{T}$ be a compactly generated triangulated category. For a functor~$H\colon\mathcal{T}^\opp\to\Ab$ the following are equivalent:
  \begin{enumerate}
    \item $H$ is cohomological, i.e.\ it sends exact triangles to exact sequences and it sends coproducts to products;
    \item $H$ is representable, i.e.\ there exists an object~$X\in\mathcal{T}$ such that~$H\cong\Hom_{\mathcal{T}}(-,X)$.
  \end{enumerate}
\end{theorem}
\begin{corollary}
  Let~$\mathcal{T}$ be a compactly generated triangulated category. Then~$\mathcal{T}$ has all coproducts.
  \begin{proof}
    Consider~$\{X_i\}_{i\in I}$ a family of objects in~$\mathcal{T}$. Then~$\prod_{i\in I}\Hom_{\mathcal{T}}(-,X_i)$ satisfies condition (1) in the Brown representability theorem, hence it must be representable by an object that satisfies the universal property for a product.
  \end{proof}
\end{corollary}
\begin{corollary}
  Let~$\mathcal{T}$ be a compactly generated triangulated category and~$\mathcal{U}$ be any triangulated category. Then for all exact functors~$F\colon\mathcal{T}\to\mathcal{U}$ the following are equivalent:
  \begin{enumerate}
    \item $F$ preserves coproducts;
    \item $F$ admits a right adjoint~$G$.
  \end{enumerate}
  \begin{proof}
    From (1) to (2) we consider any object~$X$ in~$\mathcal{U}$, then the functor
    \begin{equation}
      \Hom_{\mathcal{U}}(F(-),X)\colon\mathcal{T}^\opp\to\Ab 
    \end{equation}
    satisfies condition (1) in the Brown representability theorem, hence it can be represented as~$\Hom_{\mathcal{T}}(-,Y)$, and set~$G(X)\coloneqq Y$.

    From (2) to (1) is trivial as left adjoints preserve coproducts.
  \end{proof}
\end{corollary}
We will often take~$\mathcal{U}$ a Verdier quotient in applications.

\subsection{Bousfield localisation}
Let~$\mathcal{T}$ be a triangulated category and~$\mathcal{S}$ a thick subcategory. Its \emph{Verdier quotient}
\begin{equation}
  \mathcal{T}/\mathcal{S}\coloneqq\mathcal{T}[\{\sigma\in\Mor(\mathcal{T})\mid\cone(\sigma)\in\mathcal{S}\}^{-1}]
\end{equation}
is again a triangulated category, the quotient functor~$Q\colon\mathcal{T}\to\mathcal{S}$ is exact and its kernel~$\ker(Q)=\mathcal{S}$, where the \emph{kernel} and \emph{essential image}
\begin{equation}
  \begin{aligned}
    \ker(F)&\coloneqq\{X\in\mathcal{T}\mid F(X)=0\} \\
    \image(F)&\coloneqq\{Y\in\mathcal{U}\mid \exists X\in\mathcal{T}\colon Y\cong F(X)\}
  \end{aligned}
\end{equation}
are full triangulated subcategories.
\begin{example}
  Let~$\mathcal{A}$ be an abelian category. Then~$\derived(\mathcal{A})\coloneqq\mathbf{K}(\mathcal{A})/\mathbf{K}_\ac(\mathcal{A})$ is a triangulated category.
\end{example}
\begin{remark}
  A priori it is unclear whether a Verdier quotient~$\mathcal{T}/\mathcal{S}$ is locally small.
\end{remark}
\begin{definition}
   There is a natural transformation~$\eta\colon\identity_{\mathcal{T}}\Rightarrow L$ such that
  \begin{enumerate}
    \item $L\circ\eta\colon L\Rightarrow L^2$ is invertible, i.e.\ for all~$X$ the morphism~$L(\eta_X)\colon L(X)\to L^2(X)$ is an isomorphism;
    \item $L\circ\eta=\eta\circ L$, i.e.\ $L(\eta_X)=\eta_{L(X)}$.
  \end{enumerate}
\end{definition}
\begin{definition}
  A \emph{localisation functor} is an exact endofunctor~$\Gamma\colon\mathcal{T}\to\mathcal{T}$ for which~$\Gamma^\opp$ is a localisation functor.
\end{definition}
\begin{remark}
  The notation~$\Gamma$ is a reference both to Grothendieck's local cohomology functor and the symmetry between~$L$ and~$\Gamma$.
\end{remark}
\begin{proposition}
  For a thick subcategory~$\mathcal{S}$ the following are equivalent:
  \begin{enumerate}
    \item $\mathcal{S}\hookrightarrow\mathcal{T}$ admits a right adjoint;
    \item $Q\colon\mathcal{T}\to\mathcal{T}/\mathcal{S}$ admits a right adjoint;% TODO check these
    \item there exists a localisation functor~$L$ such that~$\ker(L)=\mathcal{S}$;
    \item there exists a colocalisation functor~$\Gamma$ such that~$\image(\Gamma)=\mathcal{S}$.
  \end{enumerate}
\end{proposition}
In this case, the following holds:
\begin{enumerate}
  \item for all objects~$X$ in~$\mathcal{T}$ there exists a functorial triangle
    \begin{equation}
      \Gamma(X)\to X\to L(X)\to
    \end{equation}
  \item $\mathcal{S}^\perp=\image(L)=\ker(\Gamma)$ and~$\prescript{\perp}{}{(}\mathcal{S}^\perp)=\mathcal{S}$ (where a priori the left-hand side could be bigger);
  \item $\Gamma$ induces a right adjoint for the inclusion~$\mathcal{S}\hookrightarrow\mathcal{T}$;
  \item $L$ induces a left adjoint for the inclusion~$\mathcal{S}^\perp\hookrightarrow\mathcal{T}$;
\end{enumerate}
\begin{remark}
  This is related to the notion of semi-orthogonal decompositions: we obtain~$\mathcal{T}=\langle\mathcal{S}^\perp,\mathcal{S}\rangle=\langle\mathcal{S},\prescript{\perp}{}{\mathcal{S}}\rangle$. A useful mnemonic (suggested by Sasha Kuznetsov) is that~$\perp$ is always in the middle.
\end{remark}
\begin{example}
  Let~$\mathcal{T}$ be a compactly generated triangulated category. Let~$\mathcal{S}=\Loc(\mathcal{C})$ be a localising subcategory generated by a set of compact objects~$\mathcal{C}$. We obtain a diagram
  \begin{equation}
    \begin{tikzcd}
      \mathcal{S}^\cc \arrow[hook]{r} \arrow[hook]{d} & \mathcal{T}^\cc \arrow[two heads]{r} \arrow[hook]{d} & \mathcal{T}^\cc/\mathcal{S}^\cc \arrow{d} \\
      \mathcal{S} \arrow[hook]{r} & \mathcal{T} \arrow[two heads]{r} & \mathcal{T}/\mathcal{S}
    \end{tikzcd}
  \end{equation}
  where~$\mathcal{S}^\cc=\Thick(\mathcal{C})$ and the induced functor~$\mathcal{T}^\cc/\mathcal{S}^\cc\to(\mathcal{T}/\mathcal{S})^\cc$ is an equivalence up to direct summands.
\end{example}

\subsection{Describing localising subcategories}
The problem that we would like to solve is the description of all localising subcategories of a compactly generated triangulated category. Some obvious questions that arise are:
\begin{enumerate}
  \item do they form a set or a proper class?
  \item how can we explicitly describe them?
\end{enumerate}

% TODO finish this last part


\section{Day 3: Stratifying of big triangulated categories}
\subsection{Rings acting on triangulated categories}
Recall that for a commutative noetherian ring~$A$ we have the associated (big) derived category~$\derived(A)\coloneqq\derived(\Mod/A)$, and to an object~$X$ in~$\derived(A)$ we have associated a support~$\supp X$ and~$\cosupp X$ which are subsets of~$\Spec A$. This way we parametrise localising subcategories.
\begin{lemma}
  Let~$X$ and~$Y$ be objects of~$\derived(A)$. Then~$\supp X\subseteq \supp Y$ if and only if~$\Loc(X)\subseteq\Loc(Y)$.
\end{lemma}

\begin{lemma}
  Let~$X$ and~$Y$ be objects of~$\derived(A)$. Then~$\cosupp X\subseteq \cosupp Y$ if and only if~$\Coloc(X)\subseteq\Coloc(Y)$.\checkthis{I didn't copy the right hand side}
\end{lemma}

\begin{corollary}
  Let~$X$ and~$Y$ be objects of~$\derived(A)$. Then~$\Ext_A^\bullet(X,Y)=0$ if and only if~$\supp X\cap\cosupp Y=\emptyset$.
\end{corollary}

Now more generally, let~$\mathcal{T}$ be a triangulated category.
\paragraph{Problem} How can we determine that for~$X$ and~$Y$ objects in~$\mathcal{T}$ we have
\begin{equation}
  \Hom_{\mathcal{T}}^\bullet(X,Y)\coloneqq\bigoplus_{n\in\mathbb{Z}}\Hom_{\mathcal{T}}(X,\Sigma^nY)=0?
\end{equation}

Fix a~$\mathbb{Z}$\dash graded commutative ring~$R=\bigoplus_{n\in\mathbb{Z}}R^n$, i.e.\ we have~$rs=(-1)^{|r||s|}sr$ for~$r$ and~$s$ homogeneous elements. This graded-commutativity enters the picture because will often take~$R$ to be a cohomology ring.
\begin{definition}
  Let~$\mathcal{T}$ be a triangulated category and~$R$ a graded commutative ring. We say that~$\mathcal{T}$ is~\emph{$R$\dash linear}, or that~\emph{$R$ acts on~$\mathcal{T}$} if there is a homomorphism of rings
  \begin{equation}
    \varphi\colon R\to\centre^\bullet(\mathcal{T})
  \end{equation}
  where~$\centre^\bullet(\mathcal{T})$ is the \emph{graded centre} of~$\mathcal{T}$, which is the graded commutative ring whose degree~$n$ piece is given by
  \begin{equation}
    \centre^n(\mathcal{T})\coloneqq\left\{ \eta\colon\identity_{\mathcal{T}}\Rightarrow\Sigma^n\mid\eta\circ\Sigma=(-1)^n\Sigma\circ\eta \right\}.
  \end{equation}
\end{definition}
So for an object~$X$ of~$\mathcal{T}$ we get a ring homomorphism
\begin{equation}
  \varphi_X\colon R\to\End_{\mathcal{T}}^\bullet(X)
\end{equation}
and~$\Hom_{\mathcal{T}}^\bullet(X,Y)$ is a graded~$R$\dash module via~$\varphi_X$ acting on the right and~$\varphi_Y$ acting on the left, whose actions coincide up to a sign.

\paragraph{Standing assumptions} From now on we take~$R$ a graded commutative noetherian ring, $\mathcal{T}$ a compactly generated triangulated category and~$\Spec R$ the set of graded prime ideals of~$R$ (which might be confusing at first for algebraic geometers).

To a graded ideal~$\mathfrak{a}$ we assign
\begin{equation}
  \VV(\mathfrak{a})\coloneqq\left\{ \mathfrak{p}\in\Spec R\mid\mathfrak{a}\subseteq\mathfrak{p} \right\}
\end{equation}
and an~$R$\dash module~$M$ is \emph{$\mathfrak{a}$\dash torsion} if~$M_{\mathfrak{p}}=0$ for all~$\mathfrak{p}\in\Spec R\setminus\VV(\mathfrak{a})$, which is equivalent to the usual definition of torsion.

A subset~$V$ is \emph{specialisation-closed} if for all~$\mathfrak{p}\subseteq\mathfrak{q}$ such that~$\mathfrak{p}\in V$ we have~$\mathfrak{q}\in V$.

\begin{definition}
  An object~$X$ of~$\mathcal{T}$ is \emph{$V$\dash torsion} if~$\Hom_{\mathcal{T}}^\bullet(C,X)_{\mathfrak{p}}=0$ for all~$\mathfrak{p}\in\Spec R\setminus V$ and all~$C\in\mathcal{T}^\cc$.

  We then set
  \begin{equation}
    \mathcal{T}_V\coloneqq\left\{ X\in\mathcal{T}\mid\text{$X$ is $V$\dash torsion} \right\}
  \end{equation}
  which is a localising subcategory as the vanishing condition uses compact objects.
\end{definition}

\begin{proposition}
  For a specialisation-closed subset~$V$ there are
  \begin{enumerate}
    \item a localisation functor~$\LL_V\colon\mathcal{T}\to\mathcal{T}$ such that~$\ker\LL_V=\mathcal{T}_V$,
    \item a colocalisation functor~$\Gamma_V\colon\mathcal{T}\to\mathcal{T}$ such that~$\image\Gamma_V=\mathcal{T}_V$,
  \end{enumerate}
  such that we obtain a \emph{localisation triangle}
  \begin{equation}
    \Gamma_V(X)\to X\to\LL_V(X)\to
  \end{equation}
  for all objects~$X$ in~$\mathcal{T}$.
\end{proposition}

For~$\mathfrak{p}\in\Spec R$ we set the \emph{localisation} of an object~$X$ of~$\mathcal{T}$ at~$\mathfrak{p}$ to be
\begin{equation}
  X_{\mathfrak{p}}\coloneqq\LL_{\ZZ(\mathfrak{p})}(X)
\end{equation}
where~$\ZZ(\mathfrak{p})\coloneqq\{\mathfrak{q}\mid\mathfrak{q}\nsubseteq\mathfrak{p}\}=\Spec R\setminus\Spec R_{\mathfrak{p}}$. The natural map~$X\to X_{\mathfrak{p}}$ obtained from the adjunction induces an isomorphism
\begin{equation}
  \Hom_{\mathcal{T}}^\bullet(C,X)_{\mathfrak{p}}\cong\Hom_{\mathcal{T}}^\bullet(C,X_{\mathfrak{p}})
\end{equation}
for all compact objects~$C$ in~$\mathcal{T}$.

\begin{definition}
  We say that an object~$X$ of~$\mathcal{T}$ is \emph{$\mathfrak{p}$\dash local} if~$X\cong X_{\mathfrak{p}}$.
\end{definition}

So now we have the notion of~$\mathfrak{p}$\dash local objects and~$\mathfrak{p}$\dash torsion objects.

\begin{definition}
  For~$\mathfrak{p}\in\Spec R$ we set the \emph{local cohomology} of an object~$X$ to be
  \begin{equation}
    \Gamma_{\mathfrak{p}}(X)\coloneqq\Gamma_{\VV(\mathfrak{p}}\circ\LL_{\ZZ(\mathfrak{p})}(X)=\Gamma_{\VV(\mathfrak{p})}(X_{\mathfrak{p}}).
  \end{equation}
  This is an idempotent functor, and~$X\in\Gamma_{\mathfrak{p}}(\mathcal{T})$ if and only if~$X$ is both~$\mathfrak{p}$\dash local and~$\mathfrak{p}$\dash torsion.
\end{definition}

\begin{remark}
  The notation is not a coincidence: it coincides with Grothendieck's local cohomology functor in the appropriate setting.
\end{remark}

\begin{lemma}
  Let~$V$ and~$W$ be specialisation-closed subsets of~$\Spec R$ such that we have~$V\setminus W=\{\mathfrak{p}\}$. Then
  \begin{equation}
    \Gamma_V\circ\Gamma_W\cong\Gamma_{\mathfrak{p}}\cong\Gamma_W\circ\Gamma_V.
  \end{equation}
\end{lemma}
This can for instance be applied to~$V=\VV(\mathfrak{p})$ and~$W=\ZZ(\mathfrak{p})$.

\subsection{Stratification of big triangulated categories}
We now have the necessary tools to introduce stratification.
\begin{definition}
  Let~$X$ be an object of~$\mathcal{T}$, with~$R$ acting on~$\mathcal{T}$. Then
  \begin{equation}
    \supp_R X\coloneqq\left\{ \mathfrak{p}\in\Spec R\mid\Gamma_{\mathfrak{p}}(X)\neq 0 \right\}
  \end{equation}
  is the \emph{support} of~$X$.
\end{definition}
\begin{remark}
  If~$X\to Y\to Z\to$ is an exact triangle, then
  \begin{equation}
    \supp_R Y\subseteq\supp_R X\cup\supp_R Y
  \end{equation}
  and
  \begin{equation}
    \supp_R(\coprod_{i\in I}X_i)=\bigcup_{i\in I}\supp_R X_i.
  \end{equation}
\end{remark}

\begin{theorem}
  For an object~$X$ of~$\mathcal{T}$ we have
  \begin{equation}
    \supp_R X=\bigcup_{C\in\mathcal{T}^\cc}\min\{\Supp_R\Hom_{\mathcal{T}}^\bullet(X,Y)\}
  \end{equation}
  where
  \begin{equation}
    \Supp_R M\coloneqq\left\{ \mathfrak{p}\in\Spec R\mid M_{\mathfrak{p}}\neq 0 \right\}
  \end{equation}
  and
  \begin{equation}
    \min\mathcal{U}\coloneqq\left\{ \mathfrak{p}\in\mathcal{U}\mid\mathfrak{q}\in\mathcal{U},\mathfrak{q}\subseteq\mathfrak{p}\Rightarrow\mathfrak{q}=\mathfrak{p} \right\}.
  \end{equation}
\end{theorem}
Hence the minimal elements determine the support: as we are working with specialisation-closed subsets, the support~$\supp_R$ only considers the minimal primes.

As we have assumed~$\mathcal{T}$ to be compactly generated we get the following corollary.
\begin{corollary}
  For all objects~$X$ in~$\mathcal{T}$ we have~$\supp_R X=\emptyset$ if and only if~$X=0$.
\end{corollary}

We now come to the main definition of the lecture series.
\begin{definition}
  The category~$\mathcal{T}$ is \emph{stratified (by the action of~$R$)} if
  \begin{enumerate}
    \item we have that
      \begin{equation}
        \Loc(X)=\Loc(\{\Gamma_{\mathfrak{p}}(X)\mid\mathfrak{p}\in\Spec R\})
      \end{equation}
      for all objects~$X$ in~$\mathcal{T}$, i.e.\ we have the \emph{local-to-global principle} which says that objects are built up from~$\mathfrak{p}$\dash local and~$\mathfrak{p}$\dash torsion information;
    \item $\Gamma_{\mathfrak{p}}(\mathcal{T})$ has no proper localising subcategories, for all~$\mathfrak{p}\in\Spec R$.
  \end{enumerate}
\end{definition}

\begin{proposition}
  Let~$\mathcal{T}$ be as above.
  \begin{enumerate}
    \item The local-to-global principle holds whenever~$\dim R<+\infty$, or when~$\mathcal{T}$ ``has a model.''
    \item Suppose that the local-to-global-principle holds, then we have bijections
      \begin{equation}
        \begin{tikzcd}
          \left\{ \text{localising subcategories of $\mathcal{T}$} \right\} \arrow[leftrightarrow]{d}{1:1} & \mathcal{S}\subseteq\mathcal{T} \arrow[mapsto]{d} \\
          \left\{ \substack{\displaystyle\text{collections of localising subcategories of~$\mathcal{T}_{\mathfrak{p}}$} \\ \displaystyle\text{for all $p\in\Spec R$}} \right\} & (\mathcal{S}\cap\Gamma_{\mathfrak{p}}(\mathcal{T}))_{\mathfrak{p}\in\Spec R}.
        \end{tikzcd}
      \end{equation}
  \end{enumerate}
\end{proposition}

\begin{example}
  \begin{enumerate}
    \item Neeman: $\derived(A)$ is stratified by~$R=A$, where~$A$ is a commutative noetherian ring.
    \item Benson--Iyengar--Krause: $\StMod(kG)$ is stratified by~$\HH^\bullet(G,k)$ (where~$G$ is a finite~$p$\dash group).
  \end{enumerate}
\end{example}

\subsection{Consequences of stratification}
\begin{theorem}
  Suppose that~$\mathcal{T}$ is stratified by~$R$. Then we have a bijection
  \begin{equation}
    \begin{aligned}
      \left\{ \text{localising subcategories of~$\mathcal{T}$} \right\} & \overset{longleftrightarrow}{1:1} \left\{ \text{subsets of $\supp_R(\mathcal{T})$} \right\} \\
      \mathcal{S}\subseteq\mathcal{T} & \longmapsto\supp_R(\mathcal{S})\coloneqq\bigcup_{X\in S}\supp_R X.
    \end{aligned}
  \end{equation}
\end{theorem}

We say~$\mathcal{T}$ is \emph{noetherian} if~$\Hom_{\mathcal{T}}^\bullet(X,Y)$ is finitely generated over~$R$ for all compact objects~$X$ and~$Y$ of~$\mathcal{T}$.
\begin{lemma}
  Let~$X$ be an object of~$\mathcal{T}$. If~$\End_{\mathcal{T}}^\bullet(X)$ is noetherian then~$\supp_R X=\VV(\mathfrak{a})$, for~$\mathfrak{a}\coloneqq\ker(R\to\End_{\mathcal{T}}^\bullet(X))$.
\end{lemma}
This allows us to classify thick subcategories of~$\mathcal{T}^\cc$!
\begin{theorem}
  Let~$\mathcal{T}$ be as above, and assume moreover that~$\mathcal{T}$ is noetherian. Then we have a bijection
  \begin{equation}
    \begin{aligned}
      \left\{ \text{thick subcategories of~$\mathcal{T}^\cc$} \right\} & \overset{\longleftrightarrow}{1:1}\left\{ \text{specialisation-closed subsets of $\Spec R$} \right\} \\
      \mathcal{S}\subseteq\mathcal{T}^\cc & \longmapsto\supp_R(\mathcal{S}).
    \end{aligned}
  \end{equation}
\end{theorem}
\begin{corollary}
  Let~$\mathcal{T}$ be as above, and assume moreover that~$\mathcal{T}$ is noetherian. Then for all compact objects~$X$ and~$Y$ of~$\mathcal{T}$ we have
  \begin{equation}
    \Supp_R\Hom_{\mathcal{T}}^\bullet(X,Y)=\supp_R X\cap\supp_R Y.
  \end{equation}
\end{corollary}

\begin{example}
  \begin{enumerate}
    \item If~$(\mathcal{T},\otimes,\mathbf{1})$ is a tensor triangulated category then~$\mathcal{T}$ is~$\End_{\mathcal{T}}^\bullet(\mathbf{1})$-linear, and we obtain a calssification of tensor ideal localising subcategories. This generalises the classification of localising subcategories to any finite group~$G$, without the assumption on the order of~$G$.
    \item Let~$A$ be a finite-dimensional~$k$\dash algebra. If we wish to stratify~$\derived(A)$ we run into problems: the center can be \emph{too small}:
      \begin{enumerate}
        \item take~$A=kG$ for~$G$ a finite group, then~$\derived(A)^\cc\cong\proj/A$, but we'd rather study~$\stmod(A)$;
        \item take~$A=kQ$ for~$Q$ an acyclic quiver, then~$\centre^\bullet(\derived(A))=k$ if~$Q$ is Dynkin, which is of no use to us.
      \end{enumerate}
    \item On the other hand, $\centre^\bullet(\mathcal{T})$ can also be \emph{too big} for meaningful calculations, in which case we will often take~$\HH^\bullet(A/k)$ as the ring action on~$\mathcal{T}$.
  \end{enumerate}
\end{example}


\section{Day 4: \texorpdfstring{$\mathbf{K}(\Inj/X)$}{K(Inj/X)} and Grothendieck duality}
\subsection{\texorpdfstring{$\derived^\bounded(X)$}{Db(X)} and compact objects}
Let~$X$ be a separated noetherian scheme. We have a chain of inclusions
\begin{equation}
  \begin{tikzcd}
    \derived^\perf(X) \arrow[hook]{r} & \derived^\bounded(\coh/X) \arrow[hook]{r} & \derived(\Qcoh/X)
  \end{tikzcd}
\end{equation}
of triangulated categories. We can relate the outer two by the following proposition:
\begin{proposition}
  The derived category~$\derived(\Qcoh/X)$ is compactly generated, with $\derived(\Qcoh/X)^\cc\cong\derived^\perf(X)$.
\end{proposition}

But for Grothendieck duality (amongst other reasons) we are mostly interested in~$\derived^\bounded(\coh/X)$, which contains noncompact objects (relative to~$\derived(\Qcoh/X)$) if~$X$ is not regular. Hence the category~$\derived(\Qcoh/X)$ is ``too small'' to have~$\derived^\bounded(\coh/X)$ as its category of compacts.

To remedy this, observe that~$\Qcoh/X$ is a Grothendieck category, hence it has enough injective objects, so we have~$\Inj/X\hookrightarrow\Qcoh/X$, where~$\Inj/X$ is closed under coproducts. Writing down the definitions for the derived category we get the following situation:
\begin{equation}
  \begin{tikzcd}
    \KKK_\ac(\Inj/X) \arrow[hook]{r} \arrow[hook]{d} & \KKK(\Inj/X) \arrow[hook]{d} \arrow[two heads]{r} \arrow{rd}{Q} & \KKK(\Inj/X)/\KKK_\ac(\Inj/X) \arrow{d}{\cong} \\
    \KKK_\ac(\Qcoh/X) \arrow[hook]{r} & \KKK(\Qcoh/X) \arrow[two heads]{r} & \derived(\Qcoh/X).
  \end{tikzcd}
\end{equation}
The following proposition shows that~$\KKK(\Inj/X)$ is the ``bigger category'' that we need, with the desired compact objects.
\begin{proposition}
  The category~$\KKK(\Inj/X)$ is compactly generated, and~$Q$ induces an equivalence~$\KKK(\Inj/X)^\cc\cong\derived^\bounded(\coh/X)$.
  \begin{proof}
    The set of compact generators corresponds to injective resolutions of coherent sheaves. Checking that these are compact boils down to studying~$\coh/X$ inside~$\Qcoh/X$, and that they are generating is proved using Baer's injectivity criterion.
  \end{proof}
\end{proposition}
\begin{theorem}
  The functor~$Q$ admits a left and right adjoint. This yields a recollement% TODO fix this
  \begin{equation}
    \begin{tikzcd}
      \SSing(\Qcoh/X) & \KKK(\Inj/X) & \derived(\Qcoh/X)
    \end{tikzcd}
  \end{equation}
  where~$\SSing(\Qcoh/X)\coloneqq\KKK_\ac(\Inj/X)$ is also known as the singularity category of~$X$.

  \begin{proof}
    Brown representability gives us the right adjoint for~$Q$, because~$\Inj/X$ is closed under coproducts.

    The proof for the left adjoint went a bit haywire and is not reproduced here.
  \end{proof}
\end{theorem}

\begin{corollary}
  A product of acyclic complexes of injectives (which in general could not be exact) is again acyclic.
\end{corollary}

\begin{remark}
  Denoting the left and right adjoint by~$Q_\lambda$ and~$Q_\rho$ respectively, we get that
  \begin{equation}
    \begin{aligned}
      \image Q_\lambda&=\SSing(\Qcoh/X)^\perp=\text{K-injective complexes}, \\
      \image Q_\rho&=\prescript{\perp}{}{\SSing(\Qcoh/X)}.
    \end{aligned}
  \end{equation}
\end{remark}

\begin{exercise}
  Based on a question by Lunts--Schn\"urer: the right adjoint~$Q_\rho$ identifies~$\derived(\Qcoh/X)$ with the full subcategory of~K-injective complexes. Are they closed under taking coproducts? (Hint: if and only if~$X$ is regular)
\end{exercise}

\begin{corollary}
  The upper row of the recollement yields (as left adjoints preserve compacts)
  \begin{equation}
    \begin{tikzcd}
      \derived^\bounded(\coh/X)/\derived^\perf(X) \arrow{d} & \derived^\bounded(\coh/X) \arrow{d}{\cong} \arrow[two heads]{l} & \derived^\perf(X) \arrow{d}{\cong} \arrow[hook]{l} \\
      \SSing(\Qcoh/X)^\cc & \KKK(\Inj/X)^\cc \arrow[two heads]{l} & \derived(\Qcoh/X)^\cc \arrow[hook]{l}
    \end{tikzcd}
  \end{equation}
  where the first vertical functor is an equivalence up to direct summands: $\SSing(\Qcoh/X)^\cc$ is karoubian but the domain is not in general.
\end{corollary}
The category~$\derived^\bounded(\coh/X)/\derived^\perf(X)$ occurs in two contexts:
\begin{enumerate}
  \item singularity categories, by Orlov,
  \item stable derived categories, by Buchweitz,
\end{enumerate}
which explains the notation.

\begin{remark}
  This works for any locally noetherian Grothendieck category, provided that~$\derived(\mathcal{A})$ is compactly generated.
\end{remark}

\begin{example}
  Take~$k$ a field of characteristic~$p$ and~$G$ a finite~$p$\dash group (in the more general context of an arbitrary finite group one has to restrict to~$\otimes$\dash ideals). Then we have a recollement
  \begin{equation}
    \begin{tikzcd}
      \StMod kG & \KKK(\Inj/kG) & \derived(kG) \\
      \Proj\HH^\bullet(G,k) & \HH^\bullet(G,k) & *
    \end{tikzcd}
  \end{equation}
  The graded commutative rings below them are the natural stratifications. We see that in this case~$\derived(kG)$ is a minimal (co)localising subcategory, hence corresponds to the unique maximal prime ideal. The category on the left is the one we are the most interested in, as it tells us the most about the representation theory of~$G$.
\end{example}

\begin{example}
  The category~$\SSing(\Qcoh/X)$ has been stratified (via tensor action) when~$X$ is locally a complete intersection, this is a result by Greg Stevenson.
\end{example}

\subsection{Grothendieck duality}
Let~$A$ be a commutative noetherian ring. Associated to~$A$ we have
\begin{equation}
  \begin{aligned}
    \Inj/A & \qquad\text{injective $A$-modules} \\
    \mathrm{Proj}/A & \qquad\text{projective $A$-modules} \\
    \Flat/A & \qquad\text{flat $A$-modules} \\
    \derived^\fin(A) & \qquad\left\{ X\in\derived(\Mod/A)\mid\bigoplus_{n\in\mathbb{Z}}\HH^n(X)\text{ finitely generated over $A$} \right\}
  \end{aligned}
\end{equation}
and we will be considering the situation
\begin{equation}
  \begin{tikzcd}
    \mathrm{mod}/A \arrow[hook]{d} \arrow[hook]{rr} & & \Mod/A \arrow[hook]{d} \\
    \derived^\bounded(\mathrm{mod}/A) \arrow{r}{\cong} & \derived^\fin(A) \arrow[hook]{r} & \derived(\Mod/A).
  \end{tikzcd}
\end{equation}
We wish to find the ``infinite completion'' of Grothendieck duality. I.e.\ suppose that~$A$ admits a dualising complex~$D_A$, which is a complex of injective objects such that
\begin{equation}
  \begin{tikzcd}
    \derived^\fin(A) \arrow{r}{\RRRHom_A(-,D_A)} & \derived^\fin(A)
  \end{tikzcd}
\end{equation}
is an equivalence. Can this be lifted to unbounded complexes?

One can show that there is a diagram % TODO fix arrows
\begin{equation}
  \begin{tikzcd}
    \KKK(\mathrm{Proj}/A) & \KKK(\Flat/A) & \KKK(\Inj/A)
  \end{tikzcd}
\end{equation}
where the right adjoint to the inclusion~$\KKK(\mathrm{Proj}/A)\hookrightarrow\KKK(\Flat/A)$ is obtained using Brown representability.
\begin{proposition}
  The category~$\KKK(\mathrm{Proj}/A)$ is compactly generated, and we obtain the diagram
  \begin{equation}
    \begin{tikzcd}
      \KKK(\mathrm{Proj}/A) \arrow{r}{\Hom_A(-,A)} & \KKK(\Mod/A) \arrow{r} & \derived(\Mod/A) \\
      \KKK(\mathrm{Proj}/A)^\cc \arrow[hook]{u} \arrow{rr}{\cong} & & \derived^\fin(A)
    \end{tikzcd}
  \end{equation}
  where the compact objects are not quite the projective resolutions of finitely generated modules, but rather their duals.
\end{proposition}
This will yield the ``infinite completion'' of Grothendieck duality (in the affine case):
\begin{theorem}[Iyengar--Krause]
  The functor~$-\otimes_AD_A\colon\KKK(\mathrm{Proj}/A)\to\KKK(\Inj/A)$ is an equivalence.
\end{theorem}
The passage to the compacts gives the relationship to the usual duality
\begin{equation}
  \begin{tikzcd}
    \KKK(\mathrm{Proj}/A)^\cc \arrow{r}{-\otimes_AD_A} \arrow{d}{\Hom_A(-,A)} & \KKK(\Inj/A)^\cc \arrow{d}{\cong} \\
    \derived^\fin(A) \arrow{r}{\RRRHom_A(-,D_A)} & \derived^\fin(A)
  \end{tikzcd}
\end{equation}
with the bottom line being Grothendieck duality in the classical sense.

\begin{remark}
  The results of the PhD thesis by Murfet generalise this to noetherian schemes.
\end{remark}

\section{Day 5: Stratifying small triangulated categories}
Whereas day 3 concerned the stratification of big triangulated categories we now focus on the story in the small setting, i.e.\ the isomorphism classes of the objects in the category actually form a set.

\subsection{Example: stratifying the bounded derived category of the Kronecker algebra}
\begin{theorem}[Beilinson]
  The object~$T\coloneqq\mathcal{O}_{\mathbb{P}_k^n}\oplus\mathcal{O}_{\mathbb{P}_k^n}(1)\oplus\ldots\oplus\mathcal{O}_{\mathbb{P}_k^n}(n)$ in~$\coh/\mathbb{P}_k^n$ is tilting, i.e.\ it induces an equivalence
  \begin{equation}
    \RRRHom(T,-)\colon\derived^\bounded(\coh/\mathbb{P}_k^n)\to\derived^\bounded(\mathrm{mod}/\Lambda_n)
  \end{equation}
  where~$\Lambda_n$ is the \emph{Beilinson algebra}~$\End(T)$, which is a finite-dimensional algebra.
\end{theorem}
The Beilinson algebra is described by the Beilinson quiver with relations, which is given by
% TODO draw picture
and relations~$x_ix_j=x_jx_i$, for all~$i,j=0,\dotsc,n$.

In the special case that~$n=1$ we get the Kronecker algebra
\begin{equation}
  \begin{bmatrix} k & k^2 \\ 0 & k \end{bmatrix} 
\end{equation}
which is moreover hereditary.

\paragraph{Problem} We wish to describe the lattice of thick subcategories of~$\derived^\bounded(\mathbb{P}_k^1)$\fixthis{refer to earlier: day 3?}.

This problem boils down to describing the indecomposables, and the category is hereditary the indecomposables are the so called ``stalk complexes'', i.e.\ up to shift we have that they are
\begin{equation}
  \left\{ \mathcal{O}_{\mathbb{P}_k^1}(i)\mid i\in\mathbb{Z} \right\}\cup\left\{ \mathcal{O}_{\mathbb{P}_k^1,p^r}\mid\text{$p$ closed in $\mathbb{P}_k^1$}, r\in\mathbb{N} \right\}.
\end{equation}
We can now set up the lattice of thick subcategories:
\begin{enumerate}
  \item Denote the set of closed points by~$\mathbb{P}_k^1(k)$ and let~$\mathcal{U}\subseteq\mathbb{P}_k^1(k)$. We associate the thick subcategory
    \begin{equation}
      \mathcal{U}\mapsto\mathcal{C}_{\mathcal{U}}\coloneqq\Thick\left( \left\{ \mathcal{O}_{\mathbb{P}_k^1,p}\mid p\in\mathcal{U} \right\} \right)
    \end{equation}
    of~$\derived^\bounded(\mathbb{P}_k^1)$ to it.
  \item Let~$i\in\mathbb{Z}$. We associate the thick subcategory
    \begin{equation}
      i\mapsto\mathcal{C}_i\coloneqq\Thick\left( \left\{ \mathcal{O}_{\mathbb{P}_k^1}(i) \right\} \right)
    \end{equation}
    of~$\derived^\bounded(\mathbb{P}_k^1)$ to it.
\end{enumerate}
These thick subcategories are ordered by inclusion, which gives us the desired lattice structure. For the thick subcategories of the first kind we moreover have that~$\mathcal{U}\subseteq\mathcal{V}$ if and only if~$\mathcal{C}_{\mathcal{U}}\subseteq\mathcal{C}_{\mathcal{V}}$. The~$\mathcal{C}_i$ on the other hand are incomparable to eachother.

This gives us the following picture:
% TODO draw picture

\subsection{Stratification of bounded derived categories of hereditary algebras}

This behaviour can be generalised to all path algebras, and in the Dynkin case (i.e.\ the underlying graph of~$Q$ is of Dynkin type~$\Delta=\mathrm{A}_n,\mathrm{D}_n$ or~$\mathrm{E}_{6,7,8}$) we get the following theorem (recall that~$\centre^\bullet(\derived^\bounded(kQ))\cong k$ so we cannot use stratification by a ring action to describe the localising subcategories).
\begin{theorem}
  There are bijections
  \begin{equation}
    \begin{tikzcd}
      \left\{ \text{localising subcategories of $\derived(\Mod/A)$} \right\} \arrow[leftrightarrow]{d}{1:1} & \mathcal{C} \arrow[mapsto]{d} \\
      \left\{ \text{thick subcategories of $\derived^\bounded(\mathrm{mod}/A)$} \right\} \arrow[leftrightarrow]{d}{1:1} & \mathcal{C}\cap\derived^\bounded(\mathrm{mod}/A) \\
      \left\{ \text{non-crossing partitions of type~$\Delta$} \right\}
    \end{tikzcd}
  \end{equation}
  where non-crossing partitions are a subset of the Weyl group~$\mathrm{W}(\Delta)$
\end{theorem}

In order to describe the second bijection in this diagram we use the fact that thick subcategories of~$\derived^\bounded(\mathrm{mod}/A)$ are all generated by exceptional objects, i.e.\ are described by~$\Thick(\{E_1,\dotsc,E_r\})$. Associated to~$E_i$ is a reflection~$\mathrm{s}_{E_i}$ in the Weyl group, and the second bijection sends~$\mathcal{D}$ to the composition of reflections~$\mathrm{s}_{E_1}\dotsm\mathrm{s}_{E_r}$. This is a result of Ingalls--Thomas.

\begin{remark}
  The theorem generalises to arbitrary quivers, if one considers the set of all thick subcategories generated by exceptional objects instead of all thick subcategories.

  In general we get the following correspondence for~$A$ a hereditary finite-dimensional $k$\dash algebra.

  \begin{center}
    \begin{tabular}{cc}
      \toprule
      $\mathcal{C}\subseteq\derived^\bounded(\mathrm{mod}/A)$ & $\mathcal{C}\cap\mathrm{mod}/A$ \\\midrule
      thick & thick \\
      admissible & having a projective generator \\
      \bottomrule
    \end{tabular}
  \end{center}

  where~$\mathcal{C}\cap\mathrm{mod}/A$ denotes those objects of~$\mathcal{C}$ concentrated in degree~0, and the left adjoint to the inclusion~$\mathcal{C}\cap\mathrm{mod}/A\hookrightarrow\mathrm{mod}/A$ furnishes a projective generator, for which we have an exceptional collection. Hence admissible subcategories correspond to subcategories generated by exceptional sequences.
\end{remark}


\subsection{Stratification of small triangulated categories}
We now focus on the local-to-global principle for small triangulated categories~$\mathcal{T}$. This is based on an arXiv preprint by Benson--Iyengar--Krause.

Assume that a graded commutative ring~$R$ acts on~$\mathcal{T}$.
\begin{proposition}
  For all~$\mathfrak{p}\in\Spec R$ there exists an exact quotient functor
  \begin{equation}
    \mathcal{T}\to\mathcal{T}_{\mathfrak{p}}:X\mapsto X_{\mathfrak{p}}
  \end{equation}
  such that
  \begin{equation}
    \Hom_{\mathcal{T}}^\bullet(X,Y)_{\mathfrak{p}}\cong\Hom_{\mathcal{T}_{\mathfrak{p}}}(X_{\mathfrak{p}},Y_{\mathfrak{p}}).
  \end{equation}
\end{proposition}
This yields the notion of \emph{$\mathfrak{p}$\dash local} objects.

Fix~$\mathfrak{a}=(r_1,\dotsc,r_n)$ a homogeneous ideal of~$R$. For any homogeneous~$r\in R$, with~$d=|r|$ and~$X$ an object of~$\mathcal{T}$ we define
\begin{equation}
  X\sslash r\coloneqq\cone(X\overset{r}\to\Sigma^dX)
\end{equation}
which is a \emph{Koszul object}. For~$\mathfrak{a}=(r_1,\dotsc,r_n)$ we then define
\begin{equation}
  X\sslash\mathfrak{a}\coloneqq X_n
\end{equation}
where~$X_0\coloneqq X$ and~$X_i\coloneqq X_{i-1}\sslash r_i$, for~$i=1,\dotsc,n$.

Of course, this definition depends on the choice of generators, and moreover cones are not functorial in general. But one can obtain the following independence result.
\begin{lemma}
  $\Thick(X\sslash{\mathfrak{a}})$ is independent of the choice of generators, as
  \begin{equation}
    \Thick(X\sslash\mathfrak{a})=\left\{ Y\in\Thick(X)\mid\End_{\mathcal{T}}^\bullet(Y)=0\forall\mathfrak{p}\nsupseteq\mathfrak{a} \right\}.
  \end{equation}
\end{lemma}
This yields the notion of \emph{$\mathfrak{a}$\dash torsion} objects.

\begin{definition}
  For a homogeneous prime ideal~$\mathfrak{p}$ we set
  \begin{equation}
    X(\mathfrak{p})\coloneqq(X\sslash\mathfrak{p})_{\mathfrak{p}}\cong X_{\mathfrak{p}}\sslash\mathfrak{p}
  \end{equation}
  (i.e.\ the quotient functor and the formation of Koszul objects commute).
\end{definition}
This yields the following local-to-global principle for thick subcategories, which is a criterion for an object to belong to a thick subcategory.
\begin{theorem}
  For a thick subcategory~$\mathcal{S}$ in~$\mathcal{T}$ and~$X\in\mathcal{T}$ the following are equivalent:
  \begin{enumerate}
    \item $X\in\mathcal{S}$;
    \item $X_{\mathfrak{p}}\in\mathcal{S}_{\mathfrak{p}}$ for all~$\mathfrak{p}\in\Spec R$;
    \item $X(\mathfrak{p})\in\mathcal{S}_{\mathfrak{p}}$ for all~$\mathfrak{p}\in\Spec R$.
  \end{enumerate}
\end{theorem}
We can now define support in this situation.
\begin{definition}
  Let~$X$ be an object of a triangulated category~$\mathcal{T}$. Then the \emph{support} of~$X$ with respect to the action of~$R$ is
  \begin{equation}
    \supp_R(X)\coloneqq\left\{ \mathfrak{p}\in\Spec R\mid X(\mathfrak{p})\neq 0 \right\}.
  \end{equation}
\end{definition}
\begin{remark}
  We can also compute the support in terms of cohomology and get the following inclusion
  \begin{equation}
    \supp_R X\subseteq\left\{ \mathfrak{p}\in\Spec R\mid\End_{\mathcal{T}}^\bullet(X)_{\mathfrak{p}}\neq 0 \right\}=\Supp_R\End_{\mathcal{T}}^\bullet(X).
  \end{equation}
  The equality holds if~$\End_{\mathcal{T}}^\bullet(X)$ is finitely generated over~$R$.
\end{remark}

For a homogeneous prime ideal~$\mathfrak{p}$ of~$R$ we set
\begin{equation}
  \Gamma_{\mathfrak{p}}(\mathcal{T})\coloneqq\left\{ X\in\mathcal{T}_{\mathfrak{p}}\mid\End_{\mathcal{T}}^\bullet(X)_{\mathfrak{q}}=0 \right\},
\end{equation}
i.e.\ we take the~$\mathfrak{p}$\dash local and~$\mathfrak{p}$\dash torsion objects, which forms a thick subcategory of~$\mathcal{T}_{\mathfrak{p}}$.

\begin{definition}
  The triangulated category~$\mathcal{T}$ is \emph{stratified by the action of~$R$} if each of the~$\Gamma_{\mathfrak{p}}(\mathcal{T})$ has no proper thick subcategories.
\end{definition}

On day 3 we had a similar result for the unbounded case. The result for small categories generalises a result from Hopkins:
\begin{example}
  For a commutative noetherian ring~$A$ we have that~$\derived^\perf(A)$ is stratified by~$R=A$.
\end{example}

\subsection{Consequences of stratification}
\begin{theorem}
  Suppose that~$\mathcal{T}$ is stratified by~$R$. Then for all objects~$X$ and~$Y$ of~$\mathcal{T}$ we have that
  \begin{enumerate}
    \item $X\in\Thick(Y)$ if and only if~$\supp_R X\subseteq\supp_R Y$;
    \item $\Hom_{\mathcal{T}}^\bullet(X,Y)=0$ if and only if~$\supp_R X\cap\supp_R Y=\emptyset$.
  \end{enumerate}
\end{theorem}
Remark that (2) in the theorem relates an asymmetric condition to a symmetric one, which is an important obstruction to having a stratification.

We can also prove a converse to this theorem:
\begin{proposition}
  Assume that~$\End_{\mathcal{T}}^\bullet(X)$ is finitely generated over~$R$ for all objects~$X$ and~$Y$ of~$\mathcal{T}$. If~$\mathcal{T}$ is not stratified then there exist objects~$X$ and~$Y$ of~$\mathcal{T}$ such that~$\supp_R X=\supp_R Y$ but~$\Thick(X)\neq\Thick(Y)$.
\end{proposition}

As a final remark: the results of day 3 depend on the so-called infinite methods, such as homotopy colimits and Brown representability, whereas today's results use the notion of cohomological functors to make things work.

\end{document}

